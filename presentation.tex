\documentclass[aspectratio=169]{beamer}
\usepackage{amsfonts}
\usepackage{faktor}
\usetheme{Copenhagen}
\setbeamertemplate{navigation symbols}{}
\setbeamertemplate{headline}{}
\begin{document}
\title{Construction of $\mathbb{Z}$}
\maketitle

\begin{frame} {Assumptions}
\begin{itemize}
    \item There exists a set $\mathbb{N}=\{0, 1, 2, ...\}$.
    \item There exists $+: \mathbb{N} \times \mathbb{N} \rightarrow \mathbb{N}$ such that $(\mathbb{N}, +)$ forms a commutative monoid with identity $0 \in \mathbb{N}$. $\footnote{We will use infix notation for $+$}$
    \item The function $succ: \mathbb{N} \rightarrow \mathbb{N^+}, n \mapsto n + 1$ is injective.
\end{itemize}
\end{frame}

\begin{frame} {Goals}
\begin{itemize}
    \item Constructing the set $\mathbb{Z}$.
    \item Defining $+:\mathbb{Z} \times \mathbb{Z} \rightarrow \mathbb{Z}$ and $\cdot: \mathbb{Z} \times \mathbb{Z} \rightarrow \mathbb{Z}$ $\footnote{We will use infix notation for $+$ and $\cdot$}$
    \item Showing $(\mathbb{Z}, +, \cdot)$ is a commutative ring with multiplicative identity $1$.
\end{itemize}
\end{frame}

\begin{frame} {Defining the set $\mathbb{Z}$}
    \begin{block}{Idea}
        We want $ \overbrace{z}^{\in \; \mathbb{Z}} \equiv \overbrace{(a, b)}^{\in \; \mathbb{N} \times \mathbb{N}} \Leftrightarrow z = a-b$.\\
        Issue: This representation is not unique. E.g: $0 = 1-1 = 2 - 2 = ....$
    \end{block}

    \begin{block}{Definition: $\sim$}
    $(a, b) \sim (c, d) : \Leftrightarrow a+d=b+c$
    \end{block}

    \begin{Lemma}
        $\sim$ is an equivalence relation
    \end{Lemma}
\end{frame}

\begin{frame} {Defining the set $\mathbb{Z}$}
    \begin{proof}
        Reflexivity:\\ $\forall (a, b) \in \mathbb{N} \times \mathbb{N}: a+b = b+a$.\\
        Symmetry:\\ $(a, b) \sim (c, d) \Rightarrow c+b = b+c\underset{(a, b) \sim (c, d)}{=}a+d=d+a \Rightarrow (c, d) \sim (a, b)$.\\
        Transitivity: \\
        Let $(a, b) \sim (c, d), (c, d) \sim (e, f)$. Then \\
        $succ^{c+d}(a+f)= \underbrace{a+d}_{= b+c}+\underbrace{c+f}_{= d+e} = b+c+d+e=succ^{c+d}(b+e)$ \\
        $\underset{\text{succ injective}}{\Rightarrow} a+f=b+e$
        $\Rightarrow (a, b) \sim (e, f)$
    \end{proof}
\end{frame}

\begin{frame} {Defining the set $\mathbb{Z}$}
\begin{block}{Definition: $\mathbb{Z}$}
    $\mathbb{Z} :=  \mathbb{N} \times \mathbb{N} / \sim \;=  \{[(a, b)] \mid a,b \in \mathbb{N}\}$
\end{block}
\end{frame}

\begin{frame}{Defining $+$}
    \begin{block} {Remark}
        $(\mathbb{N} \times \mathbb{N}, +_2)$ as direct product of $(\mathbb{N}, +)$ with itself is a semigroup.
    \end{block}

    \begin{block} {Lemma}
        $\sim$ is comptabile with $+_2$.
    \end{block}

    \begin{proof}
        Let $(a, b) \sim (a', b'), (c, d) \sim (c', d')$. Then
        $(a+c) + (b'+d') = \underbrace{(a+b')}_{=b+a'} + \underbrace{(c+d')}_{=d+c'} = (b+d) + (a'+c')$ \\
        $\Rightarrow (a, b) +_2 (c, d) = (a+c, b+d) \sim (a'+c',b'+d') = (a', b') +_2 (c', d')$
    \end{proof}
\end{frame}

\begin{frame} {Defining $+$}
    \begin{block} {Corollary: Definition of $+$}
        $[(a, b)] +_3 [(c, d)] := [(a, b) +_2 (c, d)] = [(a+c, b+d)]$ is well-defined and makes $(\mathbb{Z}, +_3)$ a semigroup.
    \end{block}

    \begin{block} {Remark}
        This gives us the usual Addition on $\mathbb{Z}$: $y = a-b, z = c - d \Rightarrow y+z = a+c - (b+d)$ $\footnote{From now on we will not distinguish between $+, +_2$ and $+_3$}$
    \end{block}

    \begin{block} {Lemma}
        $(\mathbb{Z}, +)$ is an abelian group.
    \end{block}
\end{frame}

\begin{frame} {Defining $+$}
    \begin{proof}
        Commutativity: $\forall [(a,b)], [(c, d)] \in \mathbb{Z}:$ \\
        $ [(a, b)] + [(c, d)] = [(a+c, b+d)] = [(c+a, d+b)] = [(c, d)] + [(a, b)]$ \\ 
        $\newline$ 
        Neutral Element: $\forall [(a, b)] \in \mathbb{Z}: [(a, b)] + [(0, 0)] = [(a, b)]$ \\
        $\newline$
        Inverses: $\forall [(a,b)] \in \mathbb{Z}: [(a, b)] + [(b, a)] = [\underset{\sim (0, 0)}{(a+b, b+a)}] = [(0, 0)]$
    \end{proof}
\end{frame}

\begin{frame}{Differenzdarstellung}
    \begin{block}{Definition: -}
        For $\alpha, \beta \in \mathbb{Z}$ we define:  $\alpha - \beta := \alpha + (- \beta)$
    \end{block}

    \begin{block}{Identification of $\mathbb{N}$}
        The Map $\iota: \mathbb{N} \rightarrow \mathbb{Z}, n \mapsto [(n, 0)]$ is injective and compatible with $+$.
    \end{block}

    \begin{proof}
        Injective:\\
        $[(a, 0)] = [(b, 0)] \Rightarrow a+0 = b+0 \Rightarrow a=b$\\
        $\newline$
        Compatible: $\forall a, b \in \mathbb{N}$: \\
        $\iota(a+b) = [(a+b, 0)] = [(a, 0)] + [(b, 0)] = \iota(a) + \iota(b)$
    \end{proof}
\end{frame}

\begin{frame}{Differenzdarstellung}
    \begin{block}{Identification of $\mathbb{N}$}
        We identify $\mathbb{N}$ with the isomorphic set $\iota(\mathbb{N}) \subseteq \mathbb{Z}$.
    \end{block}

    \begin{block}{Differenzdarstellung}
        We can now represent integers as $[(a, b)] = [(a, 0)] + [(0, b)] = [(a, 0)] - [(b, 0)] = a - b$
    \end{block}
\end{frame}

\begin{frame} {Definition of $\cdot$}
\end{frame}

\end{document}