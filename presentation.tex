\documentclass[aspectratio=169]{beamer}
\usepackage{amsfonts}
\usepackage{faktor}
\usetheme{Copenhagen}
\setbeamertemplate{navigation symbols}{}
\setbeamertemplate{headline}{}
\begin{document}
\title{Construction of $\mathbb{Z}$}
\maketitle

\begin{frame} {Assumptions}
\begin{itemize}
    \item There exists a set $\mathbb{N}=\{0, 1, 2, ...\}$.
    \item There exists $+: \mathbb{N} \times \mathbb{N} \rightarrow \mathbb{N}$ such that $(\mathbb{N}, +)$ forms a commutative monoid with identity $0$. $\footnote{We will use infix notation for $+$}$
    \item The function $succ: \mathbb{N} \rightarrow \mathbb{N^+}, n \mapsto n + 1$ is injective.
    \item There exists $\cdot: \mathbb{N} \times \mathbb{N} \rightarrow \mathbb{N}$ such that $(\mathbb{N}, \cdot)$ is a commutative monoid with identity $1$ 
    \item The functions $\varphi_k: \mathbb{N} \rightarrow \mathbb{N}, x \mapsto kx$ are injective for $k \in \mathbb{N}^+$
    \item There exists the usual total order $\le$ on $\mathbb{N}$
\end{itemize}
\end{frame}

\begin{frame} {Goals}
\begin{itemize}
    \item Constructing the set $\mathbb{Z}$.
    \item Defining $+:\mathbb{Z} \times \mathbb{Z} \rightarrow \mathbb{Z}$ and $\cdot: \mathbb{Z} \times \mathbb{Z} \rightarrow \mathbb{Z}$ $\footnote{We will use infix notation for $+$ and $\cdot$}$
    \item Showing $(\mathbb{Z}, +, \cdot)$ is a commutative ring with multiplicative identity $1$.
    \item Showing $\le$ is a total order on $\mathbb{Z}$
    \item Constructing the set $\mathbb{Q}$
\end{itemize}
\end{frame}

\begin{frame} {Defining the set $\mathbb{Z}$}
    \begin{block}{Idea}
        We want $ \overbrace{z}^{\in \; \mathbb{Z}} \equiv \overbrace{(a, b)}^{\in \; \mathbb{N} \times \mathbb{N}} \Leftrightarrow z = a-b$.\\
        Issue: This representation is not unique. E.g: $0 = 1-1 = 2 - 2 = ....$
    \end{block}

    \begin{block}{Definition: $\sim$}
    $(a, b) \sim (c, d) : \Leftrightarrow a+d=b+c$
    \end{block}

    \begin{Lemma}
        $\sim$ is an equivalence relation
    \end{Lemma}
\end{frame}

\begin{frame} {Defining the set $\mathbb{Z}$}
    \begin{proof}
        Reflexivity:\\ $\forall (a, b) \in \mathbb{N} \times \mathbb{N}: a+b = b+a$.\\
        Symmetry:\\ $(a, b) \sim (c, d) \Rightarrow c+b = b+c\underset{(a, b) \sim (c, d)}{=}a+d=d+a \Rightarrow (c, d) \sim (a, b)$.\\
        Transitivity: \\
        Let $(a, b) \sim (c, d), (c, d) \sim (e, f)$. Then \\
        $succ^{c+d}(a+f)= \underbrace{a+d}_{= b+c}+\underbrace{c+f}_{= d+e} = b+c+d+e=succ^{c+d}(b+e)$ \\
        $\underset{\text{succ injective}}{\Rightarrow} a+f=b+e$
        $\Rightarrow (a, b) \sim (e, f)$
    \end{proof}
\end{frame}

\begin{frame} {Defining the set $\mathbb{Z}$}
\begin{block}{Definition: $\mathbb{Z}$}
    $\mathbb{Z} :=  \mathbb{N} \times \mathbb{N} / \sim \;=  \{[(a, b)] \mid a,b \in \mathbb{N}\}$
\end{block}
\end{frame}

\begin{frame}{Defining $+$}
    \begin{block} {Remark}
        $(\mathbb{N} \times \mathbb{N}, +_2)$ as direct product of $(\mathbb{N}, +)$ with itself is a semigroup.
    \end{block}

    \begin{block} {Lemma}
        $\sim$ is comptabile with $+_2$.
    \end{block}

    \begin{proof}
        Let $(a, b) \sim (a', b'), (c, d) \sim (c', d')$. Then
        $(a+c) + (b'+d') = \underbrace{(a+b')}_{=b+a'} + \underbrace{(c+d')}_{=d+c'} = (b+d) + (a'+c')$ \\
        $\Rightarrow (a, b) +_2 (c, d) = (a+c, b+d) \sim (a'+c',b'+d') = (a', b') +_2 (c', d')$
    \end{proof}
\end{frame}

\begin{frame} {Defining $+$}
    \begin{block} {Corollary: Definition of $+$}
        $[(a, b)] +_3 [(c, d)] := [(a, b) +_2 (c, d)] = [(a+c, b+d)]$ is well-defined and makes $(\mathbb{Z}, +_3)$ a semigroup.
    \end{block}

    \begin{block} {Remark}
        This gives us the usual Addition on $\mathbb{Z}$: $y = a-b, z = c - d \Rightarrow y+z = a+c - (b+d)$ $\footnote{From now on we will not distinguish between $+, +_2$ and $+_3$}$
    \end{block}

    \begin{block} {Lemma}
        $(\mathbb{Z}, +)$ is an abelian group.
    \end{block}
\end{frame}

\begin{frame} {Defining $+$}
    \begin{proof}
        Commutativity: $\forall [(a,b)], [(c, d)] \in \mathbb{Z}:$ \\
        $ [(a, b)] + [(c, d)] = [(a+c, b+d)] = [(c+a, d+b)] = [(c, d)] + [(a, b)]$ \\ 
        $\newline$ 
        Neutral Element: $\forall [(a, b)] \in \mathbb{Z}: [(a, b)] + [(0, 0)] = [(a, b)]$ \\
        $\newline$
        Inverses: $\forall [(a,b)] \in \mathbb{Z}: [(a, b)] + [(b, a)] = [\underset{\sim (0, 0)}{(a+b, b+a)}] = [(0, 0)]$
    \end{proof}
\end{frame}

\begin{frame}{Difference representation}
    \begin{block}{Definition: -}
        For $\alpha, \beta \in \mathbb{Z}$ we define:  $\alpha - \beta := \alpha + (- \beta)$
    \end{block}

    \begin{block}{Identification of $\mathbb{N}$}
        The Map $\iota: \mathbb{N} \rightarrow \mathbb{Z}, n \mapsto [(n, 0)]$ is injective and compatible with $+$.
    \end{block}

    \begin{proof}
        Injective:\\
        $[(a, 0)] = [(b, 0)] \Rightarrow a+0 = b+0 \Rightarrow a=b$\\
        $\newline$
        Compatible: $\forall a, b \in \mathbb{N}$: \\
        $\iota(a+b) = [(a+b, 0)] = [(a, 0)] + [(b, 0)] = \iota(a) + \iota(b)$
    \end{proof}
\end{frame}

\begin{frame}{Difference representation}
    \begin{block}{Identification of $\mathbb{N}$}
        We identify $\mathbb{N}$ with the isomorphic set $\iota(\mathbb{N}) \subseteq \mathbb{Z}$.
    \end{block}

    \begin{block}{Difference representation}
        We can now represent integers as $[(a, b)] = [(a, 0)] + [(0, b)] = [(a, 0)] - [(b, 0)] = a - b$
    \end{block}
\end{frame}

\begin{frame} {Definition of $\cdot$}
    \begin{block}{Idea}
        We want $(a-b)(c-d) = ac-ad-bc+bd=ac+bd-(ad+bc)$
    \end{block}

    \begin{block}{Definition: $\cdot$}
        $[(a, b)] \cdot [(c, d)] := [(ac +bd, ad+bc)] = [(ca+db, da+cb)] = [(c, d)] \cdot [(a, b)]$
    \end{block}

    \begin{block}{$\cdot$ is well-defined}
        Let $[(a, b)]= [(a', b')]$. We have \\
        $[(a', b')] \cdot [(c, d)] = [(a'c+b'd, a'd+b'c)] = [\underset{\sim (ac, bc)}{(a'c, b'c)}]+[\underset{\sim (bd, ad)}{(b'd, a'd)]} = [(ac+bd, ad+bc)]$ \\
        By symmetry $\cdot$ is also invariant under changes of representative in the $2$nd argument.
    \end{block}
\end{frame}

\begin{frame}{Definition of $\cdot$}
    \begin{block} {Lemma}
        $(\mathbb{Z}, \cdot)$ is a commutative monoid.
    \end{block}

    \begin{proof}
        Associativity: $\forall [(a,b)], [(c,d)], [(e, f)] \in \mathbb{Z}$: \\
        $[(a, b)] \cdot ([(c, d)] \cdot [(e,f)]) = [(a,b)] \cdot [(ce+df, cf+de)]$ \\
        $=[(e(ac+bd)+f(ad+bc), f(ac+bd)+e(ad+bc))]$ \\
        $= [(ac+bd, ad+bc)] \cdot [(e, f)]$ \\
        $= ([(a,b)] \cdot [(c,d)])\cdot [(e, f)]$\\
        $\newline$
        Neutral Element: $\forall [(a,b)] \in \mathbb{Z}$:\\
        $[(a, b)] \cdot [(1, 0)] = [(a \cdot 1 + b \cdot 0, a \cdot 0 + b \cdot 1)] = [(a, b)]$
    \end{proof}
\end{frame}

\begin{frame}{$\mathbb{Z}$ is an integral domain}
    \begin{block}{Corollary}
        $\mathbb{Z}$ is a commutative ring with identity.
    \end{block}

    \begin{proof}
        $(\mathbb{Z}, +)$ is an abelian group and $(\mathbb{Z}, \cdot)$ is a commutative monoid. We also have:\\
        $\newline$
        Distributivity: $\forall [(a, b)], [(c, d)], [(e, f)]$:\\
        $[(a, b)]\cdot ([(c, d)] + [(e, f)]) = [(a(c+e) + b(d +f), a(d+f)+b(c+e))] = [(a, b)] \cdot [(c, d)] + [(a, b)] \cdot [(e, f)]$
    \end{proof}
\end{frame}

\begin{frame} {$\mathbb{Z}$ is an integral doman}
    \begin{block}{Corollay}
        $\mathbb{Z}$ is an integral domain
    \end{block}

    \begin{proof}
        Let $[(a, b)] [(c, d)] = [(0, 0)], [(a, b)] \ne [(0, 0)]$\\
        $\Rightarrow ac+bd = ad+bc$\\
        $\Rightarrow (a-b)c = (a-b)d$\\
        Assume $a > b$, so $a- b= k$ for some $k \ge 1$\\ 
        $\Rightarrow kc = kd \underset{\substack{\varphi_k: \mathbb{N} \rightarrow \mathbb{N}, x \mapsto kx \\ \text{injective for} \; k \in \mathbb{N}^+}}{\Rightarrow} c = d \Rightarrow [(c, d)] = [(0, 0)]$ \\
        $\newline$
        If $a < b$, then for $k := a-b$ the map $\varphi_{-k}: \mathbb{N} \rightarrow \mathbb{N}, x \mapsto (-k) x$ is injective.
    \end{proof}
\end{frame}

\begin{frame} {Ordering on $\mathbb{Z}$}
    \begin{block}{Definition}
        For $a, b \in \mathbb{Z}$ we define $a \le b: \Leftrightarrow b-a \in \mathbb{N}$
    \end{block}

    \begin{lemma}
        $\le$ is an order relation
    \end{lemma}
\end{frame}

\begin{frame}{Ordering on $\mathbb{Z}$}
    \begin{proof}
        Reflexivity: $\forall a \in \mathbb{Z}: a - a = 0 \in \mathbb{N}$\\
        $\newline$
        Antisymmetry: Let $a \le b, b \le a$. \\
        $\Rightarrow \exists n_1, n_2 \in \mathbb{N}: b - a = n_1, \; a-b=n_2$.\\
        $\Rightarrow n_1=b-a = -(a-b)=-n_2$\\
        $\Rightarrow n_1 = b_2 = 0 \Rightarrow a = b$\\
        $\newline$
        Transitivity: Let $a \le b, b\le c$. \\
        $\Rightarrow \exists k_1, k_2 \in \mathbb{N}: a+k_1 = b, \; b+k_2 = c$\\
        $\Rightarrow \exists k_3 \in \mathbb{N}: a+k_3 = c$ \\
        $\Rightarrow c-a \in \mathbb{N} \Rightarrow a \le c$
    \end{proof}
\end{frame}

\begin{frame} {Ordering on $\mathbb{Z}$}
    \begin{lemma}
        $\le$ is a total order.
    \end{lemma}

    \begin{proof}
        Let $a, b, c, d \in \mathbb{N}, [(a, b)] \not \le [(c, d)]$.\\
        $\Rightarrow [(c, d)] - [(a, b)] = [(b+c, a+d)] \not \in \mathbb{N}$\\
        $\Rightarrow \forall n \in \mathbb{N}: [(n, 0)] \not = [(b+c, a+d)]$\\
        $\Rightarrow \forall n \in \mathbb{N}: a+d+n \not = b+c$\\
        $\Rightarrow b+c \le a+d$\\
        $\Rightarrow a+d - (b+c) \in \mathbb{N}$\\
        $\Rightarrow [(a, b)] - [(c, d)] = [(a+d, b+c)] = [(a+d-(b+c), 0)] \in \mathbb{N}$\\
        $\Rightarrow [(c, d)] \le [(a, b)]]$
    \end{proof}
\end{frame}

\begin{frame}{Definition of $\mathbb{Q}$}
    \begin{block}{Definition}
        For $(a, b), (c, d) \in \mathbb{Z} \times (\mathbb{Z} \setminus \{0\})$ we define $(a, b) \sim (c, d): \Leftrightarrow ad=bc$  
    \end{block}

    \begin{block}{$\sim$ is an equivalence relation}
        Transitivity: $\forall (a, b) \in \mathbb{Z} \times (\mathbb{Z} \setminus \{0\}): ab=ba$\\
        Symmetry: Let $(a,b) \sim (c,d)$\\
        $\Rightarrow cb=bc=ad=da \Rightarrow (c, d) \sim (a, b)$\\
        Transitivity: Let $(a, b) \sim (c, d), (c, d) \sim (e, f)$\\
        If $c = 0$ then $a=e=0 \Rightarrow af=0=be \Rightarrow (a, b) \sim (e, f)$\\
        Else $cd \not = 0$, therefore: \\
        $cdaf=cdbe \Rightarrow af=be \Rightarrow (a, b) \sim (e, f)$  
    \end{block}
\end{frame}

\begin{frame}{Definition of $\mathbb{Q}$}
    \begin{block}{Definition: $\mathbb{Q}$}
        $\mathbb{Q} := \{[(a, b)] \mid (a, b) \in \mathbb{Z} \times (\mathbb{Z} \setminus \{0\})\}$. We define $\frac{a}{b} := [(a,b)]$.
    \end{block}
\end{frame}

\begin{frame}{Definition of $+$$(\mathbb{Q}, +)$ is an abelian group}
    \begin{block}{Definition}
        For $\frac{a}{b}, \frac{c}{d} \in \mathbb{Q}$ we define $\frac{a}{b} + \frac{c}{d} := \frac{ad+bc}{bd} = \frac{cb+ad}{db}=\frac{c}{d}+\frac{a}{b}$.
    \end{block} 

    \begin{block}{$+$ is well-defined}
        Let $\frac{a}{b} = \frac{a'}{b'}$. Then for $\frac{c}{d} \in \mathbb{Q}$:\\
        $bd(a'd+b'c) = ddba' + bb'cd = ddab' + bb'cd = b'd(ad+bc)$\\
        $\Rightarrow \frac{a'}{b'} + \frac{c}{d} = \frac{a'd+b'c}{b'd} = \frac{ad+bc}{bd} = \frac{a}{b}+\frac{c}{d}$\\ 
        $\newline$
        By Symmetry invariance under changes of the right represenative follows.
    \end{block}
\end{frame}

\begin{frame}{$(\mathbb{Q}, +)$ is an abelian group}
    \begin{block}{Proposition}
        $(\mathbb{Q}, +)$ is an abelian group
    \end{block}

    \begin{proof}
    Associativity: $\forall \frac{a}{b}, \frac{c}{d}, \frac{e}{f} \in \mathbb{Q}$: \\
    $\frac{a}{b} + \underbrace{(\frac{c}{d} + \frac{e}{f})}_{= \frac{cf+de}{df}} = \frac{adf+bcf+bde}{bdf} = \underbrace{(\frac{a}{b} + \frac{c}{d})}_{=\frac{ad+bc}{bd}} + \frac{e}{f}$\\
    $\newline$
    Neutral Element: $\forall \frac{a}{b} \in \mathbb{Q}:$ \\
    $\frac{0}{1}+\frac{a}{b}=\frac{a}{b}$ \\
    $\newline$
    Inverses: $\forall \frac{a}{b} \in \mathbb{Q}:$\\
    $\frac{a}{b}+\frac{-a}{b} = \frac{ab-ab}{ab} = \frac{0}{1}$
    \end{proof}
\end{frame}

\begin{frame}{$(\mathbb{Q} \setminus \{0\}, \cdot)$ is an abelian group}
    \begin{block}{Definition}
        For $\frac{a}{b}, \frac{c}{d} \in \mathbb{Q}$ we define: $\frac{a}{b} \cdot \frac{c}{d} := \frac{ac}{bd} = \frac{c}{d} \frac{a}{b}h$
    \end{block}

    \begin{block}{$\cdot$ is well-defined}
        Let $\frac{a}{b}=\frac{a'}{b'}$. Then for $\frac{c}{d}$: \\
        $ab'=ba' \Rightarrow ab'cd=ba'cd$\\
        $\Rightarrow \frac{a}{b} \cdot \frac{c}{d} = \frac{ac}{bd} = \frac{a'c}{b'd} = \frac{a'}{b'} \cdot \frac{c}{d}$
    \end{block}    
\end{frame}

\begin{frame}{$(\mathbb{Q} \setminus \{0\}, \cdot)$ is an abelian group}
    \begin{block}{Proposition}
        $(\mathbb{Q} \setminus \{0\}, \cdot)$ is an abelian group
    \end{block}

    \begin{proof}
        Associativity: $\forall \frac{a}{b}, \frac{c}{d}, \frac{e}{f} \in \mathbb{Q}\setminus \{0\}$: \\
        $\frac{a}{b} \cdot (\frac{c}{d} \cdot \frac{e}{f})= \frac{ace}{bdf} = (\frac{a}{b} \cdot \frac{c}{d}) \cdot \frac{e}{f}$ \\
        $\newline$
        Neutral Element: $\forall \frac{a}{b} \in \mathbb{Q}$: \\
        $\frac{a}{b} \cdot \frac{1}{1} = \frac{a}{b}$\\
        $\newline$
        Inverses: $\forall \frac{a}{b} \in \mathbb{Q}\setminus \{0\}$: \\
        $\frac{a}{b} \cdot \frac{b}{a}= \frac{ab}{ab} = \frac{1}{1}$
    \end{proof}
\end{frame}

\begin{frame}{$\mathbb{Q}$ is a field}
    \begin{block}{Lemma}
        $\mathbb{Q}$ is a field
    \end{block}

    \begin{proof}
        Since $(\mathbb{Q}, +)$ and $(\mathbb{Q} \setminus \{0\}, \cdot)$ are abelian groups, we only need to show Distributivity.\\
        Distributivity: Let $\frac{a}{b}, \frac{c}{d}, \frac{e}{f} \in \mathbb{Q}$. Then: \\
        $\frac{a}{b}\underbrace{(\frac{c}{d}+\frac{e}{f})}_{\frac{cf+de}{df}}= \frac{acf+ade}{bdf} = \frac{b(acf+dae)}{b dbf}= \frac{ac}{bd} + \frac{ae}{bf} =  \frac{a}{b}\frac{c}{d}+\frac{a}{b}\frac{e}{f}$
    \end{proof}
\end{frame}

\begin{frame}{Embedding of $\mathbb{Z}$ into $\mathbb{Q}$}
    \begin{block}{Definition}
        The Map $\iota: \mathbb{Z} \rightarrow \mathbb{Q}, z \mapsto \frac{z}{1}$ is injective and compatible with $+, \cdot$.
    \end{block}

    \begin{proof}
        Injective: Let $\frac{z}{1} = \frac{z'}{1}$. By Definition of $\sim$ we get $z=z'$.\\
        Addition: $\forall z, z' \in \mathbb{Z}: \iota(z+z')=\frac{z+z'}{1}=\frac{z}{1}+\frac{z'}{1}=\iota(z)+\iota(z')$\\
        Multiplication: $\forall z, z' \in \mathbb{Z}: \iota(zz') = \frac{zz'}{1}= \frac{z}{1}\frac{z'}{1}= \iota(z)\iota(z')$
    \end{proof}

    \begin{block}{Embedding of $\mathbb{Z}$}
        We identify $\mathbb{Z}$ with the isomorphic set $\iota(\mathbb{Z}) \subset \mathbb{Q}$
    \end{block}
\end{frame}

\end{document}